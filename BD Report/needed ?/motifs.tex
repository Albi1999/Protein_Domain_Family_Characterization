% !TEX root = template.tex
\section{Motifs}

To identify conserved short motifs within the disordered regions of our sequences, we first extracted the disordered regions from MobiDB-lite~\cite{mobidb} for the union of the sequences predicted by our models. The results, summarized in Table~\ref{tab:disordered_regions}, show that only 7 out of $83$ proteins have disordered regions annotated in MobiDB-lite.

\begin{table}[H]
\centering
\begin{tabular}{|c|c|}
\hline
\textbf{Protein ID} & \textbf{Disordered Regions} \\ \hline
P27878 & (161, 194), (405, 437) \\ \hline
P02843 & (158, 196), (407, 439) \\ \hline
P27587 & (158, 191), (399, 422) \\ \hline
P02844 & (21, 44), (165, 200), (408, 442) \\ \hline
P06607 & (401, 420) \\ \hline
Q3SZ79 & (23, 44) \\ \hline
P11602 & (471, 490) \\ \hline
\end{tabular}
\caption{Disordered regions for different protein IDs.}
\label{tab:disordered_regions}
\end{table}

Subsequently, we extracted motif patterns from ELM classes~\cite{elm} and ProSite patterns~\cite{prosite} and searched for these patterns within the previously identified disordered regions. This analysis yielded a total of $134$ matches, with an average of $19.14$ matches per sequence. For converting ProSite patterns into Python-compatible regular expressions, we used a script adapted from Stevin Wilson's PrositePatternsToPythonRegex project~\cite{stevin_wilson}.
