
%\newpage
\section{Results}

\subsection{Model Evaluation}
We evaluated the performance of PSI-BLAST and HMM-SEARCH models at both the protein and residue levels against the SwissProt database. At the protein level, both models successfully identified $82$ ground truth proteins, with only one false positive and no false negatives (Table~\ref{tab:confusion}). This indicates high precision and recall for both methods. At the residue level, the PSSM model demonstrated higher precision and F-scores compared to the HMM model, although the HMM model achieved similar recall values, particularly in detecting conserved residues (Table~\ref{tab:metrics}).

\begin{table}[H]
    \centering
    \small
    \caption{Confusion Matrices for PSI-BLAST and HMM at Protein and Residue Levels}
    \label{tab:confusion}
    \begin{tabular}{llrrrr}
        \toprule
        \textbf{Level} & \textbf{Method} & \textbf{TP} & \textbf{FP} & \textbf{FN} & \textbf{TN} \\
        \midrule
        \multirow{2}{*}{Protein} 
        & PSI-BLAST & 82 & 1 & 0 & 0 \\
        & HMM & 82 & 1 & 0 & 0 \\
        \midrule
        \multirow{2}{*}{Residue} 
        & PSI-BLAST & 23,393 & 2,247 & 1,622 & 2,352 \\
        & HMM & 23,287 & 5,578 & 1,728 & 2,280 \\
        \bottomrule
    \end{tabular}
\end{table}

\begin{table}[H]
    \centering
    \small
    \caption{Performance Metrics for PSI-BLAST and HMM at Protein and Residue Levels}
    \label{tab:metrics}
    \begin{tabular}{lcccc}
        \toprule
        \multirow{2}{*}{\textbf{Metric}} & \multicolumn{2}{c}{\textbf{Protein Level}} & \multicolumn{2}{c}{\textbf{Residue Level}} \\
        \cmidrule(lr){2-3} \cmidrule(lr){4-5}
        & PSI-BLAST & HMM & PSI-BLAST & HMM \\
        \midrule
        Precision & 0.9880 & 0.9880 & 0.9124 & 0.8068 \\
        Recall & 1.0000 & 1.0000 & 0.9352 & 0.9309 \\
        F-score & 0.9939 & 0.9939 & 0.9236 & 0.8644 \\
        Bal. Acc. & 0.5000 & 0.5000 & 0.7233 & 0.6105 \\
        MCC & 0.0000 & 0.0000 & 0.4745 & 0.2882 \\
        \bottomrule
    \end{tabular}
\end{table}

The superior performance of the PSSM model can be attributed to its ability to capture position-specific amino acid conservation patterns effectively. However, the HMM model's underperformance may reflect the challenges posed by a relatively small training dataset of $155$ sequences and the need for further optimization of its parameters.

\subsection{Taxonomy Analysis}
We visualized the taxonomic distribution of the protein family using data from UniProt~\cite{uniprot}, plotted as a phylogenetic tree (Fig.~\ref{fig:phylo-tree}). The tree revealed a strong focus within \textit{Chordata} ($n=57$), particularly in \textit{Mammalia} ($n=50$). The main clusters within mammals include \textit{Rodentia} ($n=22$) and \textit{Primates} ($n=12$). These results suggest that the domain is well-conserved across mammalian lineages, which may reflect its fundamental roles in lipid metabolism and energy regulation. The clustering of \textit{Rodentia} and \textit{Primates} further highlights the domain's relevance in key biological functions across diverse species. 

\subsection{Gene Ontology Enrichment}
Our GO enrichment analysis revealed several significant terms related to lipid metabolism and cellular processes. The most enriched function was \textit{intermediate-density lipoprotein particle remodeling} (Fig.~\ref{fig:go-wordcloud}), followed by terms such as \textit{chylomicron remodeling} and \textit{intestinal lipid catabolic process}. These terms point to the domain's involvement in lipid processing and dietary fat metabolism. Additional enriched terms included \textit{lipoprotein lipase activity}, which emphasizes the domain's functional role in lipid regulation and enzymatic activity. These findings align with previous research on the biological significance of the domain in lipid-related pathways. The hierarchical structure of enriched GO branches (Table~\ref{tab:go_terms}) further highlights its specialization in metabolic and catalytic processes.

\begin{table}[H]
    \centering
    \resizebox{1\linewidth}{!}{%
    \begin{tabular}{|l|l|c|c|c|}
        \hline
        \textbf{GO Term} & \textbf{Branch Name} & \textbf{Depth} & \textbf{Enriched} & \textbf{Score} \\
        \hline
        biological\_process & biological\_process & 0 & 216 & 2738.77 \\
        cellular process & biological\_process & 1 & 84 & 1729.83 \\
        molecular\_function & molecular\_function & 0 & 40 & 1418.64 \\
        metabolic process & cellular process & 2 & 59 & 1190.66 \\
        catalytic activity & molecular\_function & 1 & 18 & 1070.03 \\
        primary metabolic process & metabolic process & 3 & 42 & 1021.10 \\
        hydrolase activity & catalytic activity & 2 & 17 & 1008.70 \\
        hydrolase activity, acting on ester bonds & hydrolase activity & 3 & 15 & 913.04 \\
        biological regulation & biological\_process & 1 & 71 & 548.16 \\
        regulation of biological process & biological regulation & 2 & 66 & 533.08 \\
        \hline
    \end{tabular}%
    }
    \caption{Top 10 enriched GO terms and their branches}
    \label{tab:go_terms}
\end{table}

\subsection{Motif Analysis in Disordered Regions}
Our analysis of motifs within disordered regions revealed conserved patterns that may serve as functional hotspots. Out of the $83$ proteins analyzed, only seven were found to contain annotated disordered regions~\cite{mobidb} (Table~\ref{tab:disordered_regions}). From these regions, we identified $134$ conserved motifs using ELM~\cite{elm} and ProSite~\cite{prosite} patterns, with an average of $19.14$ motifs per sequence. These motifs, located within flexible regions, likely contribute to transient protein interactions and regulatory functions. These findings underscore the functional versatility of disordered regions in cellular signaling and interaction networks.

\begin{table}[H]
\centering
\begin{tabular}{|c|c|}
\hline
\textbf{Protein ID} & \textbf{Disordered Regions} \\ \hline
P27878 & (161, 194), (405, 437) \\ \hline
P02843 & (158, 196), (407, 439) \\ \hline
P27587 & (158, 191), (399, 422) \\ \hline
P02844 & (21, 44), (165, 200), (408, 442) \\ \hline
P06607 & (401, 420) \\ \hline
Q3SZ79 & (23, 44) \\ \hline
P11602 & (471, 490) \\ \hline
\end{tabular}
\caption{Disordered regions for different protein IDs.}
\label{tab:disordered_regions}
\end{table}
