% !TEX root = template.tex

\section{Results}

\subsection{Model Evaluation}
The models were evaluated on both the protein and residue levels using PSI-BLAST and HMM-SEARCH against the SwissProt database. Key findings include:
\begin{itemize}
    \item \textbf{Protein-Level Accuracy:} Both models identified $82$ ground truth proteins with only one false positive and no false negatives, indicating high precision and recall (Table~\ref{tab:confusion}).
    \item \textbf{Residue-Level Metrics:} The PSSM model slightly outperformed the HMM model, achieving higher F-scores and precision. However, the HMM model demonstrated robust recall, particularly in identifying conserved residues across sequences (Table~\ref{tab:metrics}).
\end{itemize}

\subsection{Taxonomy Analysis}
Using lineage data from UniProt, we visualized the taxonomic distribution of the family proteins. The phylogenetic tree (Fig.~\ref{fig:phylo-tree}) highlighted:
\begin{itemize}
    \item A dominant presence of proteins within the mammalian clade, especially in the order \textit{Primates}.
    \item Significant diversity observed in arthropods, suggesting evolutionary adaptation of the domain across distant species.
\end{itemize}

\subsection{Gene Ontology (GO) Enrichment}
Gene Ontology annotations revealed key functional insights:
\begin{itemize}
    \item Enriched terms primarily related to lipase activity, lipid metabolism, and structural organization, supporting the biological role of the Lipase/vitellogenin domain.
    \item Using Fisher's exact test, we identified statistically significant GO terms with p-values < 0.05 (both two-tail and right-tail). The enriched terms were visualized as a word cloud (Fig.~\ref{fig:go-wordcloud}), emphasizing their biological significance.
\end{itemize}

\subsection{Motif Analysis in Disordered Regions}
Disordered regions from MobiDB-lite were analyzed for conserved motifs. Findings include:
\begin{itemize}
    \item Only $7$ out of $83$ family proteins had annotated disordered regions (Table~\ref{tab:disordered_regions}).
    \item Motif analysis using ELM and ProSite patterns identified $134$ conserved motifs, with an average of $19.14$ motifs per sequence. This indicates potential functional hotspots within the disordered regions.
\end{itemize}