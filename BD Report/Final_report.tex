\documentclass[10pt, conference, letterpaper]{IEEEtran}

\usepackage{algorithm}
\usepackage{algorithmicx}
\usepackage{algpseudocode}
\usepackage{amsfonts}
\usepackage{amsmath}
\usepackage{amssymb}
\usepackage[ansinew]{inputenc} 
\usepackage{xcolor}
\usepackage{mathtools}
\usepackage{graphicx}
\usepackage{caption}
\usepackage{subcaption}
\usepackage{import}
\usepackage{multirow}
\usepackage[export]{adjustbox}
\usepackage{breqn}
\usepackage{mathrsfs}
\usepackage{acronym}
%\usepackage[keeplastbox]{flushend}
\usepackage{setspace}
\usepackage{bm}
\usepackage{stackengine}
\usepackage{tikz}
\usepackage[margin=1in]{geometry}
\usepackage{seqsplit}
\usepackage{booktabs}
\usepackage[table]{xcolor}
\usepackage{url}
\usepackage[backend=biber,style=numeric,citestyle=numeric]{biblatex}
\addbibresource{biblio.bib}
\usepackage{hyperref}
\usepackage{float}
\usepackage{siunitx}
\usepackage{xspace}
\usepackage{tabularx}
\usepackage{microtype}
\usepackage{array}
\usepackage{pdflscape}
\usepackage{longtable}
\usepackage{subfiles}

\hypersetup{
    colorlinks=true,
    linkcolor=blue,
    citecolor=blue,
    filecolor=magenta,
    urlcolor=cyan,
    breaklinks=true,
    pdfauthor={Your Name},
    pdftitle={Your Title},
}


\usepackage{listings}

\lstset{%
 backgroundcolor=\color[gray]{.85},
 basicstyle=\small\ttfamily,
 breaklines = true,
 keywordstyle=\color{red!75},
 columns=fullflexible,
}%

\lstdefinelanguage{BibTeX}
  {keywords={%
      @article,@book,@collectedbook,@conference,@electronic,@ieeetranbstctl,%
      @inbook,@incollectedbook,@incollection,@injournal,@inproceedings,%
      @manual,@mastersthesis,@misc,@patent,@periodical,@phdthesis,@preamble,%
      @proceedings,@standard,@string,@techreport,@unpublished%
      },
   comment=[l][\itshape]{@comment},
   sensitive=false,
  }

\usepackage{listings}

% listings settings from classicthesis package by
% Andr\'{e} Miede
\lstset{language=[LaTeX]Tex,%C++,
    keywordstyle=\color{RoyalBlue},%\bfseries,
    basicstyle=\small\ttfamily,
    %identifierstyle=\color{NavyBlue},
    commentstyle=\color{Green}\ttfamily,
    stringstyle=\rmfamily,
    numbers=none,%left,%
    numberstyle=\scriptsize,%\tiny
    stepnumber=5,
    numbersep=8pt,
    showstringspaces=false,
    breaklines=true,
    frameround=ftff,
    frame=single
    %frame=L
}

\renewcommand{\thetable}{\arabic{table}}
\renewcommand{\thesubtable}{\alph{subtable}}

\DeclareMathOperator*{\argmin}{arg\,min}
\DeclareMathOperator*{\argmax}{arg\,max}

\def\delequal{\mathrel{\ensurestackMath{\stackon[1pt]{=}{\scriptscriptstyle\Delta}}}}

\graphicspath{{./figures/}}
\setlength{\belowcaptionskip}{0mm}
\setlength{\textfloatsep}{8pt}

\newcommand{\eq}[1]{Eq.~\eqref{#1}}
\newcommand{\fig}[1]{Fig.~\ref{#1}}
\newcommand{\tab}[1]{Tab.~\ref{#1}}
\newcommand{\secref}[1]{Section~\ref{#1}}

\newcommand\MR[1]{\textcolor{blue}{#1}}
\newcommand\red[1]{\textcolor{red}{#1}}
\newcommand{\mytexttilde}{{\raise.17ex\hbox{$\scriptstyle\mathtt{\sim}$}}}

%\renewcommand{\baselinestretch}{0.98}
% \renewcommand{\bottomfraction}{0.8}
% \setlength{\abovecaptionskip}{0pt}
\setlength{\columnsep}{0.2in}

% \IEEEoverridecommandlockouts\IEEEpubid{\makebox[\columnwidth]{PUT COPYRIGHT NOTICE HERE \hfill} \hspace{\columnsep}\makebox[\columnwidth]{ }} 

\title{Functional and Structural Characterization of a protein domain family }

\author{
    \begin{tabular}{cc}
        \textbf{Alberto Calabrese} & \textbf{Marlon Helbing} \\
        \small Data Science, Department of Mathematics, University of Padova & \small Data Science, Department of Mathematics, University of Padova \\
        \small Email: \texttt{alberto.calabrese.2@studenti.unipd.it} & \small Email: \texttt{marlonjoshua.helbing@studenti.unipd.it} \\[1.5em]
        \multicolumn{2}{c}{\textbf{Lorenzo Baietti}} \\
        \multicolumn{2}{c}{\small Data Science, Department of Mathematics, University of Padova} \\
        \multicolumn{2}{c}{\small Email: \texttt{lorenzo.baietti@studenti.unipd.it}}
    \end{tabular}
}

\IEEEoverridecommandlockouts

\newcounter{remark}[section]
\newenvironment{remark}[1][]{\refstepcounter{remark}\par\medskip
   \textbf{Remark~\thesection.\theremark. #1} \rmfamily}{\medskip}

\begin{document}

\maketitle

\begin{abstract}
This study focuses on the characterization of the protein domain family associated with Pfam identifier \texttt{PF00151}, specifically the Lipase/vitellogenin domain in \textit{Homo sapiens}. We constructed a Position-Specific Scoring Matrix (PSSM) and a Hidden Markov Model (HMM) to represent the domain, evaluated them against SwissProt annotations, and analyzed their functional and structural aspects. The models effectively captured conserved sequence features, and predictions matched well with known annotations. The results emphasize the domain's critical biological roles and suggest avenues for further refinement and exploration.
\end{abstract}

% !TEX root = template.tex

\section{Introduction}
\label{sec:introduction}

The domain of a protein is an important part to model, as it can evolve, function and exist independently of the rest of the protein chain~\cite{domain_structure}. Therefore, in this work, we aim to characterize the domain of a \textit{Homo sapiens} protein that is linked to Lipase/Vitellogenin by building a sequence model starting from a single sequence and providing a functional characterization of the entire domain family. This report is structured as follows. In Section II, we describe how we built both a PSSM~\cite{psiblast} and an HMM~\cite{hmmer} model, which are then evaluated in their performance on the protein and residue level against SwissProt~\cite{uniprot} proteins annotated with the 'ground truth' Pfam domain~\cite{pfam} in Section III. Next, we looked at functional and structural properties of the entire protein family, such as taxonomy in Section IV, function in Section V, and finally motifs in Section VI.  In Section VII we report results and a conclusion is provided in Section VII.


% !TEX root = template.tex

\section{Model Building}

Given our initial data (Table~\ref{tab:protein_info}) and the representative domain sequence (Fig.~\ref{fig:sequence}), we collected $1000$ homologous sequences by means of a BLAST search~\cite{blast} against the UniProt Knowledgebase (UniProtKB)~\cite{uniprot} with an e-value threshold of $0.0001$. Next, we utilized ClustalOmega~\cite{clustalomega} to generate a multiple sequence alignment. To have a more generalizable and performant model later on, we cleaned the MSA by first removing redundant rows at a $100$\%\ threshold using JalView~\cite{jalview}, resulting in $155$ sequences, and then performing a detailed conservation analysis. In particular, for the HMM model input, we first removed columns of residues(? maybe called different) that had $90$\%\ or more gaps, which resulted in roughly removing $70$\%\ of the initially $3254$ columns. For the PSSM model input, ... (lorenzo). Given the filtered MSAs, we then proceeded to build both the HMM~\cite{hmmer} and PSSM models using HMMER-3.4~\cite{hmmer} and NCBI-BLAST+~\cite{ncbi-blast}, respectively.

\begin{table}[h]
    \centering
    \caption{Protein Domain Information}
    \label{tab:protein_info}
    \begin{tabular}{ll}
        \toprule
        \textbf{Property} & \textbf{Value} \\
        \midrule
        UniProt ID & P54315 \\
        PfamID & PF00151 \\
        Domain Position & 18-353 \\
        Organism & Homo sapiens (Human) \\
        Pfam Name & Lipase/vitellogenin \\
        \bottomrule
    \end{tabular}
\end{table}

\begin{figure}[h]
    \centering
    \caption{Domain Sequence}
    \label{fig:sequence}
    \footnotesize
    \seqsplit{KEVCYEDLGCFSDTEPWGGTAIRPLKILPWSPEKIGTRFLLYTNENPNNFQILLLSDPSTIEASNFQMDRKTRFIIHGFIDKGDESWVTDMCKKLFEVEEVNCICVDWKKGSQATYTQAANNVRVVGAQVAQMLDILLTEYSYPPSKVHLIGHSLGAHVAGEAGSKTPGLSRITGLDPVEASFESTPEEVRLDPSDADFVDVIHTDAAPLIPFLGFGTNQQMGHLDFFPNGGESMPGCKKNALSQIVDLDGIWAGTRDFVACNHLRSYKYYLESILNPDGFAAYPCTSYKSFESDKCFPCPDQGCPQMGHYADKFAGRTSEEQQKFFLNTGEASNF}
\end{figure}

% !TEX root = template.tex

\section{Model Evaluation}

To generate predictions, we used HMM-SEARCH and PSI-BLAST with default parameters. Both searches were performed against the manually curated SwissProt database (release $29.11.2024$)~\cite{uniprot}, resulting in $83$ predicted proteins with e-values $< 0.05$ and their corresponding domain locations within each protein sequence. Since HMM-SEARCH returned multiple domain hits, we only selected the one with the smallest e-value. To generate the ground truth, we collected all 82 reviewed proteins in SwissProt annotated with the given Pfam domain~\cite{pfam} utilizing the InterPro API~\cite{interpro}. We evaluated both models against the ground truth using two approaches. First, at the protein level, we verified whether predicted proteins matched the annotated ones. Second, at the residue level, for proteins present in both our predictions and the SwissProt reviewed set, we created binary vectors representing domain positions and compared them to quantify the overlap of domain boundary predictions. To assess the performance of the model, we calculated precision, recall, F-score, balanced accuracy and MCC. Results are reported in section VII.







% !TEX root = template.tex

\section{Taxonomy}

To explore the taxonomic distribution of proteins within our family, we systematically collected and analyzed lineage data for the $83$ sequences identified. Using protein identifiers from PSI-BLAST and HMM-SEARCH, we queried the UniProt API~\cite{uniprot_api} to retrieve the complete taxonomic lineage for each protein, capturing their hierarchical classification from the broadest domain (\textit{Eukaryota}) to the most specific organism. The taxonomic tree was generated using the ETE Toolkit~\cite{ete_toolkit} and visualized in the Newick tree format~\cite{newick_format}. Node sizes in the phylogenetic tree were adjusted to reflect the relative abundance of each taxonomic entity, with larger nodes indicating higher representation in the protein family. Fig.~\ref{fig:phylo-tree} illustrates the resulting phylogenetic tree, highlighting the distribution of proteins across various taxonomic groups. To further analyze taxonomic diversity, we processed the collected lineage data to calculate frequency counts at each taxonomic level. By traversing the hierarchical taxonomy, we generated a nested dictionary that revealed the occurrence patterns of each taxonomic node across the dataset. These clustering patterns underscore the evolutionary pressures maintaining the functionality of this domain across diverse taxa, providing insights into its conserved biological roles.

\begin{figure*}[h!]
    \centering
    \includegraphics[width=0.9\textwidth]{images/phylogenetic_tree_freq.png}
    \caption{Phylogenetic tree illustrating taxonomic relationships among family sequences. Node sizes reflect relative abundance, highlighting a dominant presence in mammals.}
    \label{fig:phylo-tree}
\end{figure*}


% !TEX root = template.tex
\section{Function}

First, we obtain GO annotations~\cite{gene_ontology} for both our family proteins and the entire SwissProt database. To have a more complete representation, we then expanded these GO annotations by parsing the ontology tree~\cite{obo_parser} and adding the ancestor GO terms of each initially found GO term. In order to calculate the enrichment of each GO term in our family compared to the ones found in the SwissProt database, we used Fisher's exact test~\cite{fisher_test}. We used a $2 \times 2$ contingency table where rows indicated the presence or absence of a GO term and columns differentiated between proteins within and outside our domain family (Table~\ref{tab:contingency}). 
\begin{table}[h!]
    \centering
    \small
    \begin{tabular}{lccc}
        \toprule
         & Protein in family & \multicolumn{2}{c}{Protein not in family} \\
        \\
        \midrule
        Has GO term & $a$ & \multicolumn{2}{c}{$b$} \\
        No GO term & $c$ & \multicolumn{2}{c}{$d$} \\
        \bottomrule
    \end{tabular}
    \caption{Contingency table}
    \label{tab:contingency}
\end{table}
For each GO term, we tested two hypotheses. Under the null hypothesis, the proportion of proteins annotated with that given GO term in our domain family equals the proportion in the full SwissProt dataset. We evaluated this against two alternative hypotheses: a right-tailed test to detect enrichment (higher proportion in our family than in SwissProt) and a two-tailed test to detect any significant difference in proportions (either higher or lower). The enrichment value was then calculated as:
$$
\dfrac{family\_proportion}{swissprot\_proportion}
$$
We generated a word cloud visualization using the enriched terms (where $p < 0.05$ for both $p_{\text{two-tailed}}$ and $p_{\text{right-tailed}}$ tests) weighted by their enrichment value (Fig.~\ref{fig:go-wordcloud}). Furthermore, we reported the most enriched branches of the ontology tree based on the enriched terms. For each GO term, we parsed the ontology tree~\cite{obo_parser} up to the root and added the GO term itself as an enriched child to each found ancestor. After this process, we selected only branches - which we defined as the immediate parent of a GO term, or the GO term itself in case of a root term - that had more than $2$ enriched children and a maximum depth of $3$ to filter for high-level terms. A selection of $10$ of these branches, ranked by their cumulative significance score $S = \sum -\log_{10}(p_{\text{two-tailed}})$ calculated across all child terms, can be seen in Table~\ref{tab:go_terms}.


\begin{figure*}[h!]
    \centering
    \includegraphics[width=0.8 \textwidth]{images/go_enrichment_wordcloud.png}
    \caption{Word cloud visualization of enriched GO terms. The size of each term represents its relative frequency or significance, with larger text indicating higher enrichment. Terms are related to lipoprotein metabolism, lipase activity, and various cellular processes.}
    \label{fig:go-wordcloud}
\end{figure*}

% !TEX root = template.tex
\section{Motifs}

To identify conserved short motifs within the disordered regions of our sequences, we first extracted the disordered regions from MobiDB-lite~\cite{mobidb} for the union of the sequences predicted by our models. The results, summarized in Table~\ref{tab:disordered_regions}, show that only 7 out of $83$ proteins have disordered regions annotated in MobiDB-lite.

\begin{table}[H]
\centering
\begin{tabular}{|c|c|}
\hline
\textbf{Protein ID} & \textbf{Disordered Regions} \\ \hline
P27878 & (161, 194), (405, 437) \\ \hline
P02843 & (158, 196), (407, 439) \\ \hline
P27587 & (158, 191), (399, 422) \\ \hline
P02844 & (21, 44), (165, 200), (408, 442) \\ \hline
P06607 & (401, 420) \\ \hline
Q3SZ79 & (23, 44) \\ \hline
P11602 & (471, 490) \\ \hline
\end{tabular}
\caption{Disordered regions for different protein IDs.}
\label{tab:disordered_regions}
\end{table}

Subsequently, we extracted motif patterns from ELM classes~\cite{elm} and ProSite patterns~\cite{prosite} and searched for these patterns within the previously identified disordered regions. This analysis yielded a total of $134$ matches, with an average of $19.14$ matches per sequence. For converting ProSite patterns into Python-compatible regular expressions, we used a script adapted from Stevin Wilson's PrositePatternsToPythonRegex project~\cite{stevin_wilson}.


% !TEX root = template.tex

\section{Results}

\subsection{Model Evaluation}
The models were evaluated on both the protein and residue levels using PSI-BLAST and HMM-SEARCH against the SwissProt database. Key findings include:
\begin{itemize}
    \item \textbf{Protein-Level Accuracy:} Both models identified $82$ ground truth proteins with only one false positive and no false negatives, indicating high precision and recall (Table~\ref{tab:confusion}).
    \item \textbf{Residue-Level Metrics:} The PSSM model slightly outperformed the HMM model, achieving higher F-scores and precision. However, the HMM model demonstrated robust recall, particularly in identifying conserved residues across sequences (Table~\ref{tab:metrics}).
\end{itemize}

\subsection{Taxonomy Analysis}
Using lineage data from UniProt, we visualized the taxonomic distribution of the family proteins. The phylogenetic tree (Fig.~\ref{fig:phylo-tree}) highlighted:
\begin{itemize}
    \item A dominant presence of proteins within the mammalian clade, especially in the order \textit{Primates}.
    \item Significant diversity observed in arthropods, suggesting evolutionary adaptation of the domain across distant species.
\end{itemize}

\subsection{Gene Ontology (GO) Enrichment}
Gene Ontology annotations revealed key functional insights:
\begin{itemize}
    \item Enriched terms primarily related to lipase activity, lipid metabolism, and structural organization, supporting the biological role of the Lipase/vitellogenin domain.
    \item Using Fisher's exact test, we identified statistically significant GO terms with p-values < 0.05 (both two-tail and right-tail). The enriched terms were visualized as a word cloud (Fig.~\ref{fig:go-wordcloud}), emphasizing their biological significance.
\end{itemize}

\subsection{Motif Analysis in Disordered Regions}
Disordered regions from MobiDB-lite were analyzed for conserved motifs. Findings include:
\begin{itemize}
    \item Only $7$ out of $83$ family proteins had annotated disordered regions (Table~\ref{tab:disordered_regions}).
    \item Motif analysis using ELM and ProSite patterns identified $134$ conserved motifs, with an average of $19.14$ motifs per sequence. This indicates potential functional hotspots within the disordered regions.
\end{itemize}

% !TEX root = template.tex

\section{Conclusion}
In this study, we delved into the Lipase/vitellogenin domain family, aiming to better understand its structural, functional, and evolutionary characteristics. The models we developed—PSSM and HMM—showed robust predictive capabilities, with the PSSM model particularly excelling in precision and residue-level F-scores. These models provide valuable tools for identifying this domain across diverse protein datasets.

Our taxonomy analysis revealed the evolutionary breadth of this domain, with significant representation in mammals and arthropods. This diversity underscores the adaptive significance of the Lipase/vitellogenin domain across various biological contexts, hinting at its role in lipid metabolism and storage across species.

The functional enrichment analyses added depth to our understanding of this domain's biological significance. The highlighted GO terms, related to lipase activity and lipid-related processes, align well with the known biological roles of this domain. Furthermore, our motif discovery efforts shed light on conserved patterns within disordered regions, suggesting potential functional or interaction hotspots.

Looking forward, several promising directions emerge. Refining our models with additional sequence data can further improve their performance. Experimentally validating the functional implications of the identified motifs could uncover new insights into the domain's role in cellular processes. Additionally, exploring its specific functions within underrepresented taxa could provide a more comprehensive picture of its evolutionary trajectory.

By integrating computational modeling, evolutionary analysis, and functional insights, this study offers a multifaceted view of the Lipase/vitellogenin domain, paving the way for future explorations into its biological significance and applications.


\printbibliography

\end{document}
