% !TEX root = template.tex

\section{Model Building}

Given our initial data (Table~\ref{tab:protein_info}) and the representative domain sequence (Fig.~\ref{fig:sequence}), we collected $1000$ homologous sequences by means of a BLAST search~\cite{blast} against the UniProt Knowledgebase (UniProtKB)~\cite{uniprot} with an e-value threshold of $0.0001$. Next, we utilized ClustalOmega~\cite{clustalomega} to generate a multiple sequence alignment. To have a more generalizable and performant model later on, we cleaned the MSA by first removing redundant rows at a $100$\%\ threshold using JalView~\cite{jalview}, resulting in $155$ sequences, and then performing a detailed conservation analysis. In particular, for the HMM model input, we first removed columns of residues(? maybe called different) that had $90$\%\ or more gaps, which resulted in roughly removing $70$\%\ of the initially $3254$ columns. For the PSSM model input, ... (lorenzo). Given the filtered MSAs, we then proceeded to build both the HMM~\cite{hmmer} and PSSM models using HMMER-3.4~\cite{hmmer} and NCBI-BLAST+~\cite{ncbi-blast}, respectively.

\begin{table}[h]
    \centering
    \caption{Protein Domain Information}
    \label{tab:protein_info}
    \begin{tabular}{ll}
        \toprule
        \textbf{Property} & \textbf{Value} \\
        \midrule
        UniProt ID & P54315 \\
        PfamID & PF00151 \\
        Domain Position & 18-353 \\
        Organism & Homo sapiens (Human) \\
        Pfam Name & Lipase/vitellogenin \\
        \bottomrule
    \end{tabular}
\end{table}

\begin{figure}[h]
    \centering
    \caption{Domain Sequence}
    \label{fig:sequence}
    \footnotesize
    \seqsplit{KEVCYEDLGCFSDTEPWGGTAIRPLKILPWSPEKIGTRFLLYTNENPNNFQILLLSDPSTIEASNFQMDRKTRFIIHGFIDKGDESWVTDMCKKLFEVEEVNCICVDWKKGSQATYTQAANNVRVVGAQVAQMLDILLTEYSYPPSKVHLIGHSLGAHVAGEAGSKTPGLSRITGLDPVEASFESTPEEVRLDPSDADFVDVIHTDAAPLIPFLGFGTNQQMGHLDFFPNGGESMPGCKKNALSQIVDLDGIWAGTRDFVACNHLRSYKYYLESILNPDGFAAYPCTSYKSFESDKCFPCPDQGCPQMGHYADKFAGRTSEEQQKFFLNTGEASNF}
\end{figure}